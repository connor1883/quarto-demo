% Options for packages loaded elsewhere
\PassOptionsToPackage{unicode}{hyperref}
\PassOptionsToPackage{hyphens}{url}
\PassOptionsToPackage{dvipsnames,svgnames,x11names}{xcolor}
%
\documentclass[
  letterpaper,
  DIV=11,
  numbers=noendperiod]{scrartcl}

\usepackage{amsmath,amssymb}
\usepackage{iftex}
\ifPDFTeX
  \usepackage[T1]{fontenc}
  \usepackage[utf8]{inputenc}
  \usepackage{textcomp} % provide euro and other symbols
\else % if luatex or xetex
  \usepackage{unicode-math}
  \defaultfontfeatures{Scale=MatchLowercase}
  \defaultfontfeatures[\rmfamily]{Ligatures=TeX,Scale=1}
\fi
\usepackage{lmodern}
\ifPDFTeX\else  
    % xetex/luatex font selection
\fi
% Use upquote if available, for straight quotes in verbatim environments
\IfFileExists{upquote.sty}{\usepackage{upquote}}{}
\IfFileExists{microtype.sty}{% use microtype if available
  \usepackage[]{microtype}
  \UseMicrotypeSet[protrusion]{basicmath} % disable protrusion for tt fonts
}{}
\makeatletter
\@ifundefined{KOMAClassName}{% if non-KOMA class
  \IfFileExists{parskip.sty}{%
    \usepackage{parskip}
  }{% else
    \setlength{\parindent}{0pt}
    \setlength{\parskip}{6pt plus 2pt minus 1pt}}
}{% if KOMA class
  \KOMAoptions{parskip=half}}
\makeatother
\usepackage{xcolor}
\setlength{\emergencystretch}{3em} % prevent overfull lines
\setcounter{secnumdepth}{-\maxdimen} % remove section numbering
% Make \paragraph and \subparagraph free-standing
\ifx\paragraph\undefined\else
  \let\oldparagraph\paragraph
  \renewcommand{\paragraph}[1]{\oldparagraph{#1}\mbox{}}
\fi
\ifx\subparagraph\undefined\else
  \let\oldsubparagraph\subparagraph
  \renewcommand{\subparagraph}[1]{\oldsubparagraph{#1}\mbox{}}
\fi


\providecommand{\tightlist}{%
  \setlength{\itemsep}{0pt}\setlength{\parskip}{0pt}}\usepackage{longtable,booktabs,array}
\usepackage{calc} % for calculating minipage widths
% Correct order of tables after \paragraph or \subparagraph
\usepackage{etoolbox}
\makeatletter
\patchcmd\longtable{\par}{\if@noskipsec\mbox{}\fi\par}{}{}
\makeatother
% Allow footnotes in longtable head/foot
\IfFileExists{footnotehyper.sty}{\usepackage{footnotehyper}}{\usepackage{footnote}}
\makesavenoteenv{longtable}
\usepackage{graphicx}
\makeatletter
\def\maxwidth{\ifdim\Gin@nat@width>\linewidth\linewidth\else\Gin@nat@width\fi}
\def\maxheight{\ifdim\Gin@nat@height>\textheight\textheight\else\Gin@nat@height\fi}
\makeatother
% Scale images if necessary, so that they will not overflow the page
% margins by default, and it is still possible to overwrite the defaults
% using explicit options in \includegraphics[width, height, ...]{}
\setkeys{Gin}{width=\maxwidth,height=\maxheight,keepaspectratio}
% Set default figure placement to htbp
\makeatletter
\def\fps@figure{htbp}
\makeatother

\KOMAoption{captions}{tableheading}
\makeatletter
\@ifpackageloaded{tcolorbox}{}{\usepackage[skins,breakable]{tcolorbox}}
\@ifpackageloaded{fontawesome5}{}{\usepackage{fontawesome5}}
\definecolor{quarto-callout-color}{HTML}{909090}
\definecolor{quarto-callout-note-color}{HTML}{0758E5}
\definecolor{quarto-callout-important-color}{HTML}{CC1914}
\definecolor{quarto-callout-warning-color}{HTML}{EB9113}
\definecolor{quarto-callout-tip-color}{HTML}{00A047}
\definecolor{quarto-callout-caution-color}{HTML}{FC5300}
\definecolor{quarto-callout-color-frame}{HTML}{acacac}
\definecolor{quarto-callout-note-color-frame}{HTML}{4582ec}
\definecolor{quarto-callout-important-color-frame}{HTML}{d9534f}
\definecolor{quarto-callout-warning-color-frame}{HTML}{f0ad4e}
\definecolor{quarto-callout-tip-color-frame}{HTML}{02b875}
\definecolor{quarto-callout-caution-color-frame}{HTML}{fd7e14}
\makeatother
\makeatletter
\@ifpackageloaded{caption}{}{\usepackage{caption}}
\AtBeginDocument{%
\ifdefined\contentsname
  \renewcommand*\contentsname{Table of contents}
\else
  \newcommand\contentsname{Table of contents}
\fi
\ifdefined\listfigurename
  \renewcommand*\listfigurename{List of Figures}
\else
  \newcommand\listfigurename{List of Figures}
\fi
\ifdefined\listtablename
  \renewcommand*\listtablename{List of Tables}
\else
  \newcommand\listtablename{List of Tables}
\fi
\ifdefined\figurename
  \renewcommand*\figurename{Figure}
\else
  \newcommand\figurename{Figure}
\fi
\ifdefined\tablename
  \renewcommand*\tablename{Table}
\else
  \newcommand\tablename{Table}
\fi
}
\@ifpackageloaded{float}{}{\usepackage{float}}
\floatstyle{ruled}
\@ifundefined{c@chapter}{\newfloat{codelisting}{h}{lop}}{\newfloat{codelisting}{h}{lop}[chapter]}
\floatname{codelisting}{Listing}
\newcommand*\listoflistings{\listof{codelisting}{List of Listings}}
\makeatother
\makeatletter
\makeatother
\makeatletter
\@ifpackageloaded{caption}{}{\usepackage{caption}}
\@ifpackageloaded{subcaption}{}{\usepackage{subcaption}}
\makeatother
\ifLuaTeX
  \usepackage{selnolig}  % disable illegal ligatures
\fi
\usepackage{bookmark}

\IfFileExists{xurl.sty}{\usepackage{xurl}}{} % add URL line breaks if available
\urlstyle{same} % disable monospaced font for URLs
\hypersetup{
  pdftitle={sample},
  colorlinks=true,
  linkcolor={blue},
  filecolor={Maroon},
  citecolor={Blue},
  urlcolor={Blue},
  pdfcreator={LaTeX via pandoc}}

\title{sample}
\author{}
\date{}

\begin{document}
\maketitle

\section{Notation, Definitions, and
Facts}\label{notation-definitions-and-facts}

\subsection{QUANTITATIVE VARIABLES IN
POPULATIONS}\label{quantitative-variables-in-populations}

\begin{tcolorbox}[enhanced jigsaw, colback=white, breakable, opacityback=0, leftrule=.75mm, arc=.35mm, colframe=quarto-callout-warning-color-frame, title=\textcolor{quarto-callout-warning-color}{\faExclamationTriangle}\hspace{0.5em}{Notation}, left=2mm, rightrule=.15mm, bottomtitle=1mm, coltitle=black, titlerule=0mm, bottomrule=.15mm, toptitle=1mm, colbacktitle=quarto-callout-warning-color!10!white, opacitybacktitle=0.6, toprule=.15mm]

A \textbf{\emph{finite population}} consists of \(N\) elements, labelled
\(I = 1,...,N\), where \(N\) is a real number.

\(X\) is the name of a quantitative variable (QVAR).

For \(I = 1,...,N\), \(X_I\) is the value of \(X\) for elements \(I\).

\end{tcolorbox}

\begin{tcolorbox}[enhanced jigsaw, colback=white, breakable, opacityback=0, leftrule=.75mm, arc=.35mm, colframe=quarto-callout-note-color-frame, title=\textcolor{quarto-callout-note-color}{\faInfo}\hspace{0.5em}{Definition}, left=2mm, rightrule=.15mm, bottomtitle=1mm, coltitle=black, titlerule=0mm, bottomrule=.15mm, toptitle=1mm, colbacktitle=quarto-callout-note-color!10!white, opacitybacktitle=0.6, toprule=.15mm]

The \textbf{\emph{population mean of a QVAR}} \(X\) is represented by
the symbol \(\mu_X\), and defined as

\[\mu_X = \frac{\sum_{I = 1}^N X_I}{N}\]

\end{tcolorbox}

\begin{tcolorbox}[enhanced jigsaw, colback=white, breakable, opacityback=0, leftrule=.75mm, arc=.35mm, colframe=quarto-callout-note-color-frame, title=\textcolor{quarto-callout-note-color}{\faInfo}\hspace{0.5em}{Definition}, left=2mm, rightrule=.15mm, bottomtitle=1mm, coltitle=black, titlerule=0mm, bottomrule=.15mm, toptitle=1mm, colbacktitle=quarto-callout-note-color!10!white, opacitybacktitle=0.6, toprule=.15mm]

A \textbf{\emph{deviation from the population mean of a QVAR}} \(X\)
\textbf{\emph{for a given element}} is the difference between the value
of \(X\) for the given element and the population mean of \(X\).

The deviation from the population mean of \(X\) for element \(I\) is
represented by the symbol \(D_{X_I}\):

\[D_{X_I} = X_I - \mu_X\]

\end{tcolorbox}

\begin{tcolorbox}[enhanced jigsaw, colback=white, breakable, opacityback=0, leftrule=.75mm, arc=.35mm, colframe=quarto-callout-tip-color-frame, title=\textcolor{quarto-callout-tip-color}{\faLightbulb}\hspace{0.5em}{Fact}, left=2mm, rightrule=.15mm, bottomtitle=1mm, coltitle=black, titlerule=0mm, bottomrule=.15mm, toptitle=1mm, colbacktitle=quarto-callout-tip-color!10!white, opacitybacktitle=0.6, toprule=.15mm]

For any finite population and any QVAR \(X\), if you find the deviation
from the population mean of \(X\) for every element in the population,
and then add up all those deviations, you will find that the sum is
zero.

\end{tcolorbox}

To put that more succinctly: for any finite population and any
quantitative variable \(X\), the sum of the deviations from the means is
always equal to 0.

This is true no matter how many elements there are in the population and
no matter what the values of \(X\) are for the elements.

More formally: Given a finite population consisting of any number \(N\)
of elements, and given any values \(X_1,\,X_2,\,...,\,X_N\) of a QVAR
\(X\),

\[\sum_{I=1}^N D_{X_I} = 0\]

or equivalently,

\[\sum_{I = 1}^N \left( X_I - \mu_X \right) = 0\]

\begin{tcolorbox}[enhanced jigsaw, colback=white, breakable, opacityback=0, leftrule=.75mm, arc=.35mm, colframe=quarto-callout-tip-color-frame, title=\textcolor{quarto-callout-tip-color}{\faLightbulb}\hspace{0.5em}{Proof of this fact}, left=2mm, rightrule=.15mm, bottomtitle=1mm, coltitle=black, titlerule=0mm, bottomrule=.15mm, toptitle=1mm, colbacktitle=quarto-callout-tip-color!10!white, opacitybacktitle=0.6, toprule=.15mm]

We can prove this equality through several steps of algebra. Every step
is just a rearrangement of terms.

\[\begin{array}{l}
\sum_{I = 1}^N D_{X_I} = \sum_{I = 1}^N \left( X_I - \mu_X \right) \\
\,\,\,\,\,\,\,\,\,\,\,\,\,\,\, = \sum_{I = 1}^N X_I - \sum_{I = 1}^N \mu_X \\
\,\,\,\,\,\,\,\,\,\,\,\,\,\,\, = \sum_{I = 1}^N X_I - N \mu_X \\
\,\,\,\,\,\,\,\,\,\,\,\,\,\,\, = N \left( \frac{\sum_{I = 1}^N X_I}{N} \right) - N \mu_X \\
\,\,\,\,\,\,\,\,\,\,\,\,\,\,\, = N \mu_X - N \mu_X \\
\,\,\,\,\,\,\,\,\,\,\,\,\,\,\, = 0
\end{array}\]

\end{tcolorbox}

\begin{tcolorbox}[enhanced jigsaw, colback=white, breakable, opacityback=0, leftrule=.75mm, arc=.35mm, colframe=quarto-callout-note-color-frame, title=\textcolor{quarto-callout-note-color}{\faInfo}\hspace{0.5em}{Definition}, left=2mm, rightrule=.15mm, bottomtitle=1mm, coltitle=black, titlerule=0mm, bottomrule=.15mm, toptitle=1mm, colbacktitle=quarto-callout-note-color!10!white, opacitybacktitle=0.6, toprule=.15mm]

The \textbf{\emph{population mean absolute deviation}} \textbf{\emph{of
a QVAR}} \(X\) is represented by the symbol \(MAD_X\), and defined as

\[MAD_X = \frac{\sum_{I = 1}^N \left| X_I - \mu_X \right|}{N}\]

\end{tcolorbox}

\begin{tcolorbox}[enhanced jigsaw, colback=white, breakable, opacityback=0, leftrule=.75mm, arc=.35mm, colframe=quarto-callout-note-color-frame, title=\textcolor{quarto-callout-note-color}{\faInfo}\hspace{0.5em}{Definition}, left=2mm, rightrule=.15mm, bottomtitle=1mm, coltitle=black, titlerule=0mm, bottomrule=.15mm, toptitle=1mm, colbacktitle=quarto-callout-note-color!10!white, opacitybacktitle=0.6, toprule=.15mm]

The \textbf{\emph{population variance}} \textbf{\emph{of a QVAR}} \(X\)
is represented by the symbol \(\sigma_X^2\), and defined as

\[\sigma_X^2 = \frac{\sum_{I = 1}^N \left( X_I - \mu_X \right)^2}{N}\]

\end{tcolorbox}

\begin{tcolorbox}[enhanced jigsaw, colback=white, breakable, opacityback=0, leftrule=.75mm, arc=.35mm, colframe=quarto-callout-note-color-frame, title=\textcolor{quarto-callout-note-color}{\faInfo}\hspace{0.5em}{Definition}, left=2mm, rightrule=.15mm, bottomtitle=1mm, coltitle=black, titlerule=0mm, bottomrule=.15mm, toptitle=1mm, colbacktitle=quarto-callout-note-color!10!white, opacitybacktitle=0.6, toprule=.15mm]

The \textbf{\emph{population standard deviation}} \textbf{\emph{of a
QVAR}} \(X\) is represented by the symbol \(\sigma_X\), and defined as

\[\sigma_X = \sqrt{\sigma_X^2}\]

\end{tcolorbox}

\subsection{QUANTITATIVE VARIABLES IN
SAMPLES}\label{quantitative-variables-in-samples}

The notation we use for samples is different from the notation we use
for populations.

For some concepts, the definitions for samples are identical to the
definitions for populations, and the only differences are in the
notation.

For other concepts, the differences are not only in the notation---the
definitions for samples are similar, but not identical, to the
definitions for populations.

\begin{tcolorbox}[enhanced jigsaw, colback=white, breakable, opacityback=0, leftrule=.75mm, arc=.35mm, colframe=quarto-callout-warning-color-frame, title=\textcolor{quarto-callout-warning-color}{\faExclamationTriangle}\hspace{0.5em}{Notation}, left=2mm, rightrule=.15mm, bottomtitle=1mm, coltitle=black, titlerule=0mm, bottomrule=.15mm, toptitle=1mm, colbacktitle=quarto-callout-warning-color!10!white, opacitybacktitle=0.6, toprule=.15mm]

A \textbf{\emph{sample}} consists of \(n\) observations, labelled
\(i = 1,...,n\).

\(X\) is the name of a quantitative variable.

For \(i = 1,...,n\), \(x_i\) is the value of \(X\) for observation
\(i\).

\end{tcolorbox}

\begin{tcolorbox}[enhanced jigsaw, colback=white, breakable, opacityback=0, leftrule=.75mm, arc=.35mm, colframe=quarto-callout-note-color-frame, title=\textcolor{quarto-callout-note-color}{\faInfo}\hspace{0.5em}{Definition: The sample mean of a QVAR}, left=2mm, rightrule=.15mm, bottomtitle=1mm, coltitle=black, titlerule=0mm, bottomrule=.15mm, toptitle=1mm, colbacktitle=quarto-callout-note-color!10!white, opacitybacktitle=0.6, toprule=.15mm]

The \textbf{\emph{sample mean of a QVAR}} \(X\) is represented by the
symbol \(\bar{x}\), and defined as

\[\bar{x} = \frac{\sum_{i = 1}^n x_i}{n}\]

\end{tcolorbox}

\begin{tcolorbox}[enhanced jigsaw, colback=white, breakable, opacityback=0, leftrule=.75mm, arc=.35mm, colframe=quarto-callout-note-color-frame, title=\textcolor{quarto-callout-note-color}{\faInfo}\hspace{0.5em}{Definition: A deviation from the sample mean of a QVAR for a given
observation}, left=2mm, rightrule=.15mm, bottomtitle=1mm, coltitle=black, titlerule=0mm, bottomrule=.15mm, toptitle=1mm, colbacktitle=quarto-callout-note-color!10!white, opacitybacktitle=0.6, toprule=.15mm]

A \textbf{\emph{deviation from the sample mean of a QVAR}} \(X\)
\textbf{\emph{for a given observation}} is the difference between the
value of \(X\) for the given observation and the population mean of
\(X\). The deviation from the sample mean of \(Q\) for observation \(i\)
is represented by the symbol \(d_{X_i}\):

\[d_{X_i} = x_i - \bar{x}\]

\end{tcolorbox}

\begin{tcolorbox}[enhanced jigsaw, colback=white, breakable, opacityback=0, leftrule=.75mm, arc=.35mm, colframe=quarto-callout-tip-color-frame, title=\textcolor{quarto-callout-tip-color}{\faLightbulb}\hspace{0.5em}{Fact}, left=2mm, rightrule=.15mm, bottomtitle=1mm, coltitle=black, titlerule=0mm, bottomrule=.15mm, toptitle=1mm, colbacktitle=quarto-callout-tip-color!10!white, opacitybacktitle=0.6, toprule=.15mm]

For every sample and any QVAR \(X\), if you find the deviation from the
sample mean of \(X\) for every observation in the sample, and then add
up all those deviations, you will find that the sum is zero.

\end{tcolorbox}

To put that more succinctly: for any sample and any quantitative
variable \(X\), the sum of the deviations from the means is always equal
to 0.

This is true no matter how many observations there are in the sample and
no matter what the values of \(X\) are for the observations.

More formally: Given a sample consisting of any number \(n\) of
observations, and given any values \(x_1,\,x_2,\,...,\,x_n\) of a QVAR
\(X\),

\[\sum_{i = 1}^n d_{X_i} = 0\]

\begin{tcolorbox}[enhanced jigsaw, colback=white, breakable, opacityback=0, leftrule=.75mm, arc=.35mm, colframe=quarto-callout-tip-color-frame, title=\textcolor{quarto-callout-tip-color}{\faLightbulb}\hspace{0.5em}{Proof of this fact}, left=2mm, rightrule=.15mm, bottomtitle=1mm, coltitle=black, titlerule=0mm, bottomrule=.15mm, toptitle=1mm, colbacktitle=quarto-callout-tip-color!10!white, opacitybacktitle=0.6, toprule=.15mm]

We can prove this equality through several steps of algebra. Every step
is just a rearrangement of terms.

\[\begin{array}{l}
\sum_{i = 1}^n d_{X_i} = \sum_{i = 1}^n \left( x_i - \bar{x} \right) \\
\,\,\,\,\,\,\,\,\,\,\,\,\,\,\, = \sum_{i = 1}^n x_i - \sum_{i = 1}^n \bar{x} \\
\,\,\,\,\,\,\,\,\,\,\,\,\,\,\, = \sum_{i = 1}^n x_i - n \bar{x} \\
\,\,\,\,\,\,\,\,\,\,\,\,\,\,\, = n \left( \frac{\sum_{i = 1}^n x_i}{n} \right) - n \bar{x} \\
\,\,\,\,\,\,\,\,\,\,\,\,\,\,\, = n \bar{x} - n \bar{x} \\
\,\,\,\,\,\,\,\,\,\,\,\,\,\,\, = 0
\end{array}\]

\end{tcolorbox}

\begin{tcolorbox}[enhanced jigsaw, colback=white, breakable, opacityback=0, leftrule=.75mm, arc=.35mm, colframe=quarto-callout-note-color-frame, title=\textcolor{quarto-callout-note-color}{\faInfo}\hspace{0.5em}{Definition: The sample mean absolute deviation of a QVAR}, left=2mm, rightrule=.15mm, bottomtitle=1mm, coltitle=black, titlerule=0mm, bottomrule=.15mm, toptitle=1mm, colbacktitle=quarto-callout-note-color!10!white, opacitybacktitle=0.6, toprule=.15mm]

The \textbf{\emph{sample mean absolute deviation}} \textbf{\emph{of a
QVAR}} \(X\) is represented by the symbol \(mad_X\), and defined as

\[mad_X = \frac{\sum_{i = 1}^n \left| x_i - \bar{x} \right|}{n - 1}\]

\end{tcolorbox}

\begin{tcolorbox}[enhanced jigsaw, colback=white, breakable, opacityback=0, leftrule=.75mm, arc=.35mm, colframe=quarto-callout-note-color-frame, title=\textcolor{quarto-callout-note-color}{\faInfo}\hspace{0.5em}{Definition: The sample variance of a QVAR}, left=2mm, rightrule=.15mm, bottomtitle=1mm, coltitle=black, titlerule=0mm, bottomrule=.15mm, toptitle=1mm, colbacktitle=quarto-callout-note-color!10!white, opacitybacktitle=0.6, toprule=.15mm]

The \textbf{\emph{sample variance}} \textbf{\emph{of a QVAR}} \(X\) is
represented by the symbol \(s_X^2\), and defined as

\[s_X^2 = \frac{\sum_{i = 1}^n \left( x_i - \bar{x} \right)^2}{n - 1}\]

\end{tcolorbox}

\begin{tcolorbox}[enhanced jigsaw, colback=white, breakable, opacityback=0, leftrule=.75mm, arc=.35mm, colframe=quarto-callout-note-color-frame, title=\textcolor{quarto-callout-note-color}{\faInfo}\hspace{0.5em}{Definition: The sample standard deviation of a QVAR}, left=2mm, rightrule=.15mm, bottomtitle=1mm, coltitle=black, titlerule=0mm, bottomrule=.15mm, toptitle=1mm, colbacktitle=quarto-callout-note-color!10!white, opacitybacktitle=0.6, toprule=.15mm]

The \textbf{\emph{sample standard deviation}} \textbf{\emph{of a QVAR}}
\(X\) is represented by the symbol \(s_X\), and defined as

\[s_X = \sqrt{s_X^2}\]

\end{tcolorbox}

Edits to this document

!{[}sample.pdf{]}



\end{document}
